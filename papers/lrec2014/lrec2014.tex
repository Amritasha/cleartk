\documentclass[10pt, a4paper]{article}
\usepackage{lrec2006}
\usepackage{graphicx}

\title{ClearTK 2.0: Lessons Learned}

\name{Author1, Author2, Author3}

\address{ Affiliation1, Affiliation2, Affiliation3 \\
               Address1, Address2, Address3 \\
               author1@xxx.yy, author2@zzz.edu, author3@hhh.com\\}


\abstract{Each article must include an abstract of 150 to 200 words in Times 9 pt with interlinear spacing of 10 pt.
 The heading Abstract should be centred, font Times 10 bold. This short abstract will also be used for printing a Booklet of Abstracts 
containing the abstracts of all papers presented at the Conference. \\ \newline \Keywords{keyword A, keyword B, keyword C}}



\begin{document}

\maketitleabstract

\section{Paper}
Each manuscript should be submitted on white \textbf{A4 paper.} The fully justified text should be formatted in two parallel columns, each 8.25 cm wide, and separated by a space of 0.63 cm. Left, right, and bottom margins should be 1.9 cm. and the top margin 2.5 cm. The font for the main body of the text should be Times 10 pt with interlinear spacing of 12 pt.
\textbf{Articles must be between 4 and 8 pages in length}, regardless of the mode of presentation (oral or poster).

\section{General instructions}

Each paper is allocated between \underline{\textbf{a mi\-ni\-mum of four}} \textbf{\underline{and a maximum of eight pages} } including figures. \newline The unprotected PDF files will appear on the CD-ROM directly as received. Do not print the page number.

\section{Page numbering}

\textbf{Please do not include page numbers in your article.} The definitive page numbering of articles published in the proceedings will be decided by the organizing committee of the LREC.

\section{Headings/Level 1 Headings}

Headings should be capitalised in the same way as the main title, and centred within the column. The font used is Times 12 bold. There should also be a space of 12 pt between the title and the preceding section, and a space of 3 pt between the title and the text following it.

\subsection{Level 2 Headings}

The format for level 2 headings is basically the same as for level 1 Headings. However, the font is Times 11, and the heading is justified to the left of the column.

\subsubsection{Level 3 Headings}

The format for level 3 headings is the same as for level 2 headings, except that the font is Times 10, and there should be no space left between the heading and the text.

%\subsubsection{Example of a sub-subsection with a long heading that will occupy two lines}
%
%Yet another example of a sub-subsection. Yet another example of a sub-subsection. Yet another example of a sub-subsection. Yet another example of a sub-subsection. Yet another example of a sub-subsection.

\section{References in the text}

All references within the text should be placed in parentheses containing the author's surname followed by a comma before the date of publication \cite{Martin-90}. If the sentence already includes the author's name, then it is only necessary to put the date in parentheses: \newcite{Martin-90}. When several authors are cited, those references should be separated with a semicolon: \cite{Martin-90,CastorPollux-92}. When the reference has more than three authors, only cite the name of the first author followed by et al. (e.g. \cite{Superman-Batman-Catwoman-Spiderman-00}).

\section{Figures \& Tables}
\subsection{Figures}

All figures should be centred and clearly distinguishable. They should never be drawn by hand, and the lines must be very dark in order to ensure a high-quality printed version. Figures should be numbered in the text, and have a caption in Times 10 pt underneath. A space must be left between each figure and its respective caption.
Example of a figure enclosed in a box.

\begin{figure}[h]
\begin{center}
%\fbox{\parbox{6cm}{
%This is a figure with a caption.}}
\caption{The caption of the figure.}
\label{fig.1}
\end{center}
\end{figure}

Figure and caption should always appear together on the same page. Large figures can be centred, using a full page.
%NB: an example of large figures is missing.

\subsection{Tables}

The instructions for tables are the same as for figures (see previous section).
Example:
%Two types of tables are distinguished: in-column and big tables that don't fit in the columns.
%\subsection{In-column tables}
%An example of an in-column table is presented here.
%
\begin{table}[h]
 \begin{center}
\begin{tabular}{|l|l|}

      \hline
      Level&Tools\\
      \hline\hline
      Morphology & Pitrat Analyser\\
      Syntax & LFG Analyser (C-Structure)\\
      Semantics & LFG F-Structures + Sowa's\\
      & Conceptual Graphs\\
      \hline

\end{tabular}
\caption{The caption of the table}
 \end{center}
\end{table}

%\subsection{Big tables}
%
%An example of a big table which extends beyond the column and will
%float in the next page.
%
% \begin{table*}[ht]
% \begin{center}
% \begin{tabular}{|l|l|}
%
%       \hline
%       Level&Tools\\
%       \hline\hline
%       Morphology & Pitrat Analyser\\
%       Syntax & LFG Analyser (C-Structure)\\
%       Semantics & LFG F-Structures + Sowa's Conceptual Graphs  \\
%       \hline
%
% \end{tabular}
% \caption{The caption of the big table}
% \end{center}
% \end{table*}
%

\section{Footnotes}

Footnotes are indicated within the text by a number in superscript\footnote{They should be in Times 9, and appear at the bottom of the same page as their corresponding number. Footnotes should also be separated from the rest of the text by a horizontal line 5 cm long.}.

\section{Copyrights}

The Lan\-gua\-ge Re\-sour\-ce and Evalua\-tion Con\-fe\-rence (LREC) proceedings are published by the European Language Resources Association (ELRA). They include different media that may be used (i.e. hardcopy, CD-ROM, Internet-based/Web, etc.).

ELRA's policy is to acquire copyright for all LREC contributions. In assigning your copyright, you are not forfeiting your right to use your contribution elsewhere. This you may do without seeking permission and is subject only to normal acknowledgement to the LREC proceedings.

\section{Conclusion}

Your submission of a finalized contribution for inclusion in the LREC proceedings automatically assigns the above-mentioned copyright to ELRA.
proceedings.

\section{Acknowledgements}

Place all acknowledgements (including those concerning research grants and funding) in a separate section at the end of the article.

\section{References}
Bibliographical references should be listed in alphabetical order at the end of the article. The title of the section, "References", ``References'', should be a level 1 heading. The first line of each bibliographical reference should be justified to the left of the column, and the rest of the entry should be indented by 0.35 cm.

The following examples (of fictitious references) illustrate the basic format required for articles in conference proceedings, books, journals articles, Ph.D. theses, and chapters of books respectively:
\cite{Martin-90},  \cite{Chercheur-94},  \cite{CastorPollux-92}  \cite{Zavatta-92},  \cite{Grandchercheur-83}.

%\nocite{*}

\bibliographystyle{lrec2006}
\bibliography{lrec2014}

\end{document}

