\documentclass[final]{beamer}
\mode<presentation>
{
  \usetheme{Cleartk}
}
\usepackage{color}
\usepackage{times}
\usepackage{amsmath,amssymb}
\usepackage{sfmath} % for sans serif math fonts; wget http://dtrx.de/od/tex/sfmath.sty
\usepackage[english]{babel}
\usepackage[latin1]{inputenc}
\usepackage[size=custom,height=91.44,width=60.96,scale=1.0]{beamerposter}
\usepackage{booktabs,array}
\usepackage{listings}
\usepackage{xspace}
\usepackage{fp}
\usepackage{ifthen}

\listfiles
\newcommand*{\signstream}{SignStream\texttrademark\xspace}

\graphicspath{{/u/figures/}}

% Display a grid to help align images
%\beamertemplategridbackground[1cm]

\title{\Huge ClearTK 2.0: Design Patterns for Machine Learning in UIMA\\[0.2ex]}

\author{Steven Bethard \inst{1}, Philip Ogren \inst{2}, Lee Becker\inst{2}}
\institute[] % (optional, but mostly needed)
{
  \inst{1}%
  University of Alabama at Birmingham, Birmingham, AL, USA
  \\
  \inst{2}%
  University of Colorado Boulder, Boulder, CO, USA
}

\date[May. 29th, 2014]{May. 29th, 2014}

\begin{document}
\begin{frame}{} 
\vspace{-1cm}
\begin{columns}[t]
  %%%%%%%%%%%%%%%%%%%%%%%%%%%%%%%%%%%%%%%%%%%%%%%%%%%%%%%%%%%%%%%%%%%%%%%%%%%%%%%%%%%%%%%%%%%%%%%%%%%%
  %%%%%%%%%%%%%%%%%%%%%%%%%%%%%%%%%%%%%%%%%%%%%%%%%%%%%%%%%%%%%%%%%%%%%%%%%%%%%%%%%%%%%%%%%%%%%%%%%%%%
  \begin{column}{.45\linewidth}
    
    \begin{block}{Overview}
      \begin{itemize}
      \item ClearTK makes statistical machine learning within the UIMA framework more manageable by providing:

        \begin{itemize}
        \item A common interface and wrappers for ML libraries such as: SVMlight, LIBSVM, LIBLINEAR, OpenNLP MaxEnt and Mallet
        \item A rich feature extraction library
        \item Infrastructure for using and evaluating ML classifiers
        \item A variety of wrappers for existing NLP libraries (Stanford, CLEAR Parser, etc\ldots)
        \end{itemize}

      \item We reflect on 5 years of code-base evolution to:
        \begin{itemize}
        \item Reflect on best practices for training statistical NLP components in UIMA
        \item Discuss design patterns that make developing annotators more accessible to new users
        \end{itemize}
      \end{itemize}
    \end{block}
    
    \begin{block}{Annotators should be conceptually simple}
      \begin{itemize}
      \item 
        \begin{itemize}
        \item 
        \end{itemize}
      \item 
      \item 
        \begin{itemize}
        \item 
        \item 
        \item 
        \end{itemize}
      \end{itemize}

      \begin{center}
        \includegraphics[width=.2\linewidth]{images/camera0}
        \,
        \includegraphics[width=.2\linewidth]{images/camera1}
        \,
        \includegraphics[width=.2\linewidth]{images/camera2}
      \end{center}
    \end{block}

    \begin{block}{Pipelines should look like pipelines}
      \begin{itemize}
      \item 
      \item 
      \end{itemize}
    \end{block}
    
    \begin{block}{Collection readers should be minimal}
      \begin{itemize}
      \item 
      \end{itemize}
      
      \vskip1ex
      \centering
      \includegraphics[width=.75\linewidth]{images/dai}
      \vskip2ex
      \includegraphics[width=.45\linewidth]{images/dai-search}
      \,
      \includegraphics[width=.45\linewidth]{images/dai-results}


    \end{block}
  \end{column}

  %%%%%%%%%%%%%%%%%%%%%%%%%%%%%%%%%%%%%%%%%%%%%%%%%%%%%%%%%%%%%%%%%%%%%%%%%%%%%%%%%%%%%%%%%%%%%%%%%%%%
  % Begin second column
  %%%%%%%%%%%%%%%%%%%%%%%%%%%%%%%%%%%%%%%%%%%%%%%%%%%%%%%%%%%%%%%%%%%%%%%%%%%%%%%%%%%%%%%%%%%%%%%%%%%%

  \begin{column}{.45\linewidth}
    \begin{block}{Code should be type system agnostic}
    \end{block}

    \begin{block}{Modules should match natural subsets}
    \end{block}

    \begin{block}{Users need help past the UIMA overhead}
    \end{block}
    
   
  \end{column}

\end{columns}
\vfill
\end{frame}

\end{document}


%%%%%%%%%%%%%%%%%%%%%%%%%%%%%%%%%%%%%%%%%%%%%%%%%%%%%%%%%%%%%%%%%%%%%%%%%%%%%%%%%%%%%%%%%%%%%%%%%%%%
%%% Local Variables: 
%%% mode: latex
%%% TeX-PDF-mode: t
