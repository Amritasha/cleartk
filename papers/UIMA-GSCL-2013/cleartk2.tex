\documentclass{llncs}

\begin{document}

\title{ClearTK 2.0: Lessons learned developing a machine learning framework for UIMA}
\titlerunning{ClearTK 2.0}

\author{Steven Bethard\inst{1} \and Philip Ogren\inst{2} \and Lee Becker\inst{3}}
\authorrunning{Steven Bethard et al.}

\institute{%
University of Colorado Boulder, Boulder, CO 80309, USA \\
\email{steven.behard@colorado.edu}
\and
Oracle America, Broomfield, CO 80021, USA \\
\email{philip@ogren.info}
\and
Avaya Labs, Superior, CO 80027, USA \\
\email{lee.becker@gmail.com}}

\maketitle

\begin{abstract}
TODO

\keywords{UIMA, machine learning, natural language processing}
\end{abstract}

\section{Introduction}


\section{Lesson: Annotators should look like annotators}

CleartkAnnotator is just a JCasAnnotator

Chunking is just a utility object for use in a JCasAnnotator

Features like TF-IDF are in the Annotator, not in the encoder


\section{Lesson: Pipelines should look like pipelines}

New evaluation's train and test methods

Trainable extractors for TF-IDF etc. (not encoders)


\section{Lesson: CollectionReaders should do almost nothing}

URICollectionReader


\section{Lesson: Modules should group classes by function}
not organized by type system

cleartk-type-system

cleartk-corpus

cleartk-feature


\section{Lesson: Type-system-agnostic code requires interfaces}

Philip's blog post

weaknesses of OpenNLP approach (e.g., assumes pos is an attribute of token)

ClearNLP work


\section{Lesson: Users need help to get past the UIMA overhead}

Write the reader and eval, let the student feature-engineer


\section{Discussion}


\end{document}
